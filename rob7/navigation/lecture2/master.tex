\documentclass{article}


\usepackage{import}
\usepackage{pdfpages}
\usepackage{transparent}
\usepackage{xcolor}

% Remove paragraph indentation.
\setlength{\parindent}{0pt}


\newcommand{\incfig}[2][1]{%
	\def\svgwidth{#1\columnwidth}
	\import{./figures/}{#2.pdf_tex}
}


\pdfsuppresswarningpagegroup=1

\title{Robot Navigation: Lecture 2 \\
	\large Litterature notes}
\author{Victor Risager}
\date{\today}



\begin{document}
\maketitle

\section{Sensor classification}
Two important functional axes:
\begin{itemize}
	\item Proprioceptive or Exteroceptive
	\item Passive or active
\end{itemize}

\textit{Proprioceptive} sensors measures internal values of the robot, e.g. motor speeds, wheel loads joint angles, internal forces, battery voltage etc. 

\textit{Exteroceptive} sensors measures external information, i.e. from the environment. E.g. a Lidar 


\textit{Passive} sensors measures ambient environmental energy entering the sensor. E.g. temperature, humididty, microphones, etc. 

\textit{Active} Sensors can mange controlled interactions with the environment. These include wheel quadrature encoders, ultra sonic sensors and Lidar. 

\subsection{Performance}
\begin{itemize}
	\item Dynamic Range: Spread between lower and upper limits of the input values, during normal sensor operation. Measured in decibels $dB$ and computed as  $10 \cdot log[\frac{max}{min}]$. 
	\item Resolution
	\item Lineariry
	\item Bandwidth
\end{itemize}

\newpage


\section{Sensors for Mobile Robots}
There are 4 layers of information compression:


TODO: make slash snippet context based
\begin{enumerate}
	\item Raw data
	\item Features (corners, size, etc.)
	\item Objects
	\item Places or situations
\end{enumerate}

Cognitive systems have to interpret situation based on limited information andonly partially available informations. This is a probability problem ~ probabilistic reasoning

Wheel encoders are used for odometry, (Odometry is position estimation based on sensor information) 


\subsection{Accelerometer}
We calculate the acceleration in accelerometers with:
$F = \ddot{x} + c \overline{x} + kx $

where: $F = m \alpha$

and  $\alpha = \frac{Kx}{m}$


\subsection{IMU}
The IMU is sensitive to noise in the accelorometer and the gyroscope. 
It has drift. 
Residual gravity vector results in a quadratic error in position


\subsection{GPS} 
It uses trilatteration.  
The sattelite gives you only the position. The distance is not given. The distance is $c \cdot t$. We know the speed of the electromagnetic waves, i.e. the speed of light c. The position gives more like a gaussian distribution, which is used to estimate the position. 

\subsubsection{dGPS}
We set up a base station with a known position, which is used to correct for position errors in moving objects. 

\subsection{TOF sensors}
Time Of Flight sensors. Two different types:
\begin{enumerate}
	\item Continous wave
	\item Pulse wave
\end{enumerate}

the distance is given as 
\[
d = c \cdot t
\] 

where, $d$ is the distance,  $c$ is the speed of light, and $t$ is time it took for the travel.


\subsubsection{Laser Range Sensors}
We can use phase shift measurement. 
Measured in pico seconds ($1 \times 10^{-12}$ )



\subsection{In Situ Sensor performance}
\begin{itemize}
	\item Systematic errors
	\item Random errors
	\item Precission
	\item Accuracy
	\item Resolution
\end{itemize}


We use a PDF to represent the possible sensor values. 
We use the mean value $\mu$ to represent the mean, where  $\mu = E[X]$
Sensors can be independent. E.g. a temperature sensor and an IMU.

Whenever we can, we use the gaussian to represent the values. However this is often a simplification, which does not represent the realworld that well.

We use Variance and Covariance to represent the level of dependence. (Off. diagonal values in the covariance matrix)

\section{Paper Reading}
Papers will be presented and discussed in week 40
2 - 4 people in the group.

15 minute presentation, 5 minutes of questions. (RG)


Excel file for choosing papers. 

Deadline for selecting: 17 september

\section{Exercises}
Compass calibration. Turn the tiago robot around, and plot the values.

Take the values and point move them to 0.
Use the maximum and minimum values of the values to write the equation to move it into the origin. Then we need to reshape it from an oval to a sphere. Multiply it by the scaling factor. 



We will get the projects on monday.
We will get the projects on monday



\end{document}





