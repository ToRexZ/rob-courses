\documentclass[a4paper]{article}

\usepackage[utf8]{inputenc}
\usepackage[T1]{fontenc}
\usepackage{textcomp}
\usepackage[english]{babel}
\usepackage{amsmath, amssymb}
\usepackage{import}
\usepackage{pdfpages}
\usepackage{transparent}
\usepackage{xcolor}

% Remove paragraph indentation.
\setlength{\parindent}{0pt}

% Figure support
\usepackage{xifthen}
\pdfminorversion=7

\newcommand{\incfig}[2][1]{%
    \def\svgwidth{#1\columnwidth}
    \import{./figures/}{#2.pdf_tex}
}

\pdfsuppresswarningpagegroup=1

\title{Robot Navigation: Lecture 3 \\
	\large Odometry Sytems, Localisation}
\author{Victor Risager}

\begin{document}
\maketitle
\section{Intro}
2 different types of Odometry.

\begin{itemize}
	\item Vision based
	\item Wheel odometry
\end{itemize}

Note: We will leave it at the open loop today.

\section{Recap}
Sensor classification
\begin{itemize}
	\item Exteroceptive
	\item Proprioceptive 
\end{itemize}

Different sensors for robotics.

\vspace{5pt}
\hrule
\vspace{5pt}


\section{Odometry}
We look at 3 different coordinate system

\begin{itemize}
	\item World
	\item Robot body 
	\item Camera
\end{itemize}

\subsection{Wheel odometry}
Some encoders are absolute. \\
Some are quadrature encoders.
	- It only gives pulses


Today we will derive the differential drive robot.

\subsection{Differential-Drive Robots}

We assume that we have no slippage.

The next position of the robot is:
$
p = \begin{bmatrix}
x' \\
y' \\
\theta'
\end{bmatrix}
$

The movement of the robot wheel centers, are relative to the robot center.  

$\frac{d}{dt}x_r = \omega_r \cdot r \cdot h$ 
$\frac{d}{dt}x_l = \omega_l \cdot r \cdot h$ 
where $h$ is the the heading vector, and r is the radius of the robot.

The equations of motion is 



$\dot{x} = \omega_s \cdot h$


$\dot{h} = w_d \cdot R_{\frac{\pi}{2}} \cdot h$

We control the angular velocity of the robot itself.
from that we can solve for the two wheel velocities.


We can use the first order euler method to solve the differential equations of motion.
we discretizese the timeline.
In the slides, they use the 2'nd order method. (numerical solver)

\textbf{Proporties}
Wheel odometry drits, because the wheels slips. You don't know the exact radius of the wheels, which can be affected over time, by e.g. rubber destruction, loss of airpressure. 

Numerical error, where the error grows. 


\subsection{Alternative odometry}
\begin{itemize}
	\item Visual 
	\item Laser-based 
	\item Radar-based
	\item Auditory
	\item $\cdots$
	\item Combinations of above
\end{itemize}

\subsection{Visual Odometry}
The idea of visual odometry is that:
\begin{itemize}
	\item Identify POI (Points of Interest)
\end{itemize}

\subsubsection{The workings of a camera}
\begin{figure}[ht]
    \centering
    \incfig[1]{camera}
    \caption{The inner workings of a camera}
    \label{fig:camera}
\end{figure}


We can use triangles to compute the projection of the point into the sensor, and thus the pixel.
This gives a single dimension of positional accuracy, thus the radial distance. 

In a stereo camera, the distance between the two cameras are called the base $b$.
Having two pinholes, we can then compute the triangles of based on the position of the point of interest. 


With a stereo camera, we get local $x,y,z$ coordinates


% \begin{figure}[ht]
%    \centering
%    \incfig[1]{cameraframefig}
%    \caption{cameraFrameFig}
%    \label{fig:cameraframefig}
%\end{figure}

\section{Probabilistic map generation}

$E[\hat{x}_{n+1}] = f(x_n, u_n) + f_x E[\hat{x_n} - x_n] + f_u E[\hat{u_n} - u_n]$





\end{document}
