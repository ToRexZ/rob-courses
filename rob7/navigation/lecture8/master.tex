\documentclass[a4paper]{article}

\usepackage[utf8]{inputenc}
\usepackage[T1]{fontenc}
\usepackage{textcomp}
\usepackage[english]{babel}
\usepackage{amsmath, amssymb}
\usepackage{import}
\usepackage{pdfpages}
\usepackage{transparent}
\usepackage{xcolor}
\usepackage{framed}

% Remove paragraph indentation.
\setlength{\parindent}{0pt}

% Figure support
\usepackage{xifthen}
\pdfminorversion=7

\newcommand{\incfig}[2][1]{%
    \def\svgwidth{#1\columnwidth}
    \import{./figures/}{#2.pdf_tex}
}

\pdfsuppresswarningpagegroup=1

\title{Advanced Robot Navigation: Lecture 8 \\
	\large Lecture Notes}
\author{Victor Risager}

\begin{document}
\maketitle

\section{Recap}
Bayes filter branch out into particle filter and kalman filter.
EKF local linear approximation.

These filters are used for localisation.

\subsection{Map building}
\textbf{Landmark:} a landmark is a distinguishable feature in the environment. Camera $\rightarrow$  spot in a certain color
Leads into \textbf{SLAM} which can brach out into: 
\begin{itemize}
	\item EKF, include landmarks in state $ \rightarrow  $ landmarks$^2$
	\item PF SLAM $ \rightarrow  $ FastSLAM $ \rightarrow  $ Linear scaling
\end{itemize}

\textbf{Action model $ \leftarrow $ mobility model} which is applied to close a time gap. Other times we do a perception update, where we use a sensor model, which incorporates information 

Only update the perceptions if there is a new measurement ready.
You can use action updates and perception updates independently.

\subsection{Random walk model}
\begin{equation}
	x_{n+1} = x_n + v_n 
\end{equation}
where $ v_n $ is some measurement noise.
A very imprecise model 

if your odometry malfuncitons, then you cannot just use your motion model in the action update. 

\section{Path Planning}
Initial assumptions:
\begin{itemize}
	\item Holonomic
	\item Point robot
\end{itemize}

We will apply Minkowski-sum later on.

\subsection{Holonomic robots}
A car is non-holonomic, because the number of global degrees of freedom, are higher than the free configuration degrees of freedom. 

A train is holonomic, because we have reduced the global degrees of freedom. 

you "expand" the obstacle to confine the width of the robot and make that our new obstacle. 

If we dont do the conservative approximation of the robot by taking e.g. a circle of the robot.

\subsection{C-Obstacles}
How do we map the workspace obstacles to the C-space?

We define an anchor point on the robot, which is the point that abstracts the robot, and that point will be outside the minkowsky sum.
we check if:
\begin{equation}
p + r \in O
\end{equation}
where $ P $ is the translation of the  $ r $ vector in the Robot.

Then 
\begin{equation}
\exists o \in O, P +r = o \rightarrow p = o-r
\end{equation}

where
\begin{equation}
	C = \{P = o + r | o \in O, r \in -R \}
\end{equation}

matlab: Polyshape which gives a polygonal shape based on a set of coordinates. The minkowsky sum is a 3'rd party library. The minkowsky sum is computed with the best possible heading, and is therefore not as conservative.

Discretisising the headings and recomputing the minkowsky sum, gives a set of maps which combined on top of each other, gives a way to search position and heading.

The star algorithm produces the same result.





\end{document}
