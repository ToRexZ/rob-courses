\documentclass[a4paper]{article}

\usepackage[utf8]{inputenc}
\usepackage[T1]{fontenc}
\usepackage{textcomp}
\usepackage[english]{babel}
\usepackage{amsmath, amssymb}
\usepackage{import}
\usepackage{pdfpages}
\usepackage{transparent}
\usepackage{xcolor}
\usepackage{framed}
\usepackage[margin=2.5cm]{geometry}

% Remove paragraph indentation.
\setlength{\parindent}{0pt}

% Figure support
\usepackage{xifthen}
\pdfminorversion=7

\newcommand{\incfig}[2][1]{%
    \def\svgwidth{#1\columnwidth}
    \import{./figures/}{#2.pdf_tex}
}

\pdfsuppresswarningpagegroup=1

\title{Model Predictive Control\\
	\large Litterature Notes}
\author{Victor Risager}

\begin{document}
\maketitle

\section{A basic formulation}
We assume that state values cannot be measured, thus we need an observer. So we use state estimates 
$ \hat{x}(k|k) $ of $ x(k|k) $ indicating that it is a measurement that is based on all states up until time $ k $ \\
It is based on $ u(k-1) $ and not $ u(k) $ since that input has not been determined yet. \\
$ \hat{u}(k+i|k) $ denotes the future values at time $ k + i $ on input $ u $ which is assumed on time $ k $. This means that i is some horizon, and  $ u(k+j|k) , j = 0,1, \ldots, i-1 $ is the inputs at each time step.


we have the cost function:

\begin{align} \label{eq:costfunction}
V &= \sum_{i=H_w}^{H_p}{|| \hat{z}(k+i|k) - r(k+i|k)||_{Q(i)}^{2} } + \sum_{i = 0}^{H_u - 1}{|| \Delta \hat{u}(k+i|k)||_{R(i)}^{2} }
\end{align}

where $ r(k+i|k) $ is a reference trajectory and $ \hat{z}(k+i|k) $ is the controlled outputs. The prediction horizon has length $ H_p $ but $ H_w $ indicates the prediciton window, which determines when to start penalizing. If $ H_w > 1 $ then we only penalize from that point forward, as there may be some delay between control inputs and effects. $ H_u $ is the control horizon, where $ H_u \leq H_p $ and future control differences between $ \Delta \hat{u}(k+i|k) = 0 $ and $ \Delta \hat{u}(k+i|k) $ where $ i > H_u $. Note that the cost function in (\ref{eq:costfunction}) only penalizes changes in $ u $ and not  $ u $ itself. The matrices $ Q(i) $ and $ R(i) $ are weights and both positive semidefinite $ (\cdot) \geq 0$ 

\section{Constraints}
There are different constraints, which are assumed to hold over the entire control- and prediction horizon.

\begin{align}
	E \hspace{3pt} vec( \Delta \hat{u}(k|k), \ldots, \Delta \hat{u}(k+H_u - 1|k), 1) &\leq vec(0) \label{eq:constraint1} \\ 
	F \hspace{3pt} vec( \Delta \hat{u}(k|k), \ldots, \hat{u}(k+H_u - 1|k), 1) &\leq vec(0) \label{eq:constraint2}\\
	G \hspace{3pt} vec( \hat{z}(k+H_w | k) , \ldots, \hat{z}(k+H_p|k), 1) &\leq vec(0) \label{eq:constraint3}
\end{align}

where $ E, F, $ and $ G $ are matrices of suitable dimensions. (\ref{eq:constraint1}) can be used to represent actuator slew rate (change of input of an actuator), actuator ranges (\ref{eq:constraint2}), and control variable constraints on $ z $ based on (\ref{eq:constraint3}) 

\end{document}
