\documentclass[a4paper]{article}

\usepackage[utf8]{inputenc}
\usepackage[T1]{fontenc}
\usepackage{textcomp}
\usepackage[english]{babel}
\usepackage{amsmath, amssymb}
\usepackage{import}
\usepackage{pdfpages}
\usepackage{transparent}
\usepackage{xcolor}
\usepackage{framed}

% Remove paragraph indentation.
\setlength{\parindent}{0pt}

% Figure support
\usepackage{xifthen}
\pdfminorversion=7

\newcommand{\incfig}[2][1]{%
    \def\svgwidth{#1\columnwidth}
    \import{./figures/}{#2.pdf_tex}
}

\pdfsuppresswarningpagegroup=1

\title{Robot Mobility: Lecture 3 \\
	\large Lecture Notes}
\author{Victor Risager}

\begin{document}
\maketitle
\section{Stability}

The linear system $ \Sigma $:
\begin{equation}
\dot{x} = Ax + Bu
\end{equation}
\begin{equation}
y = Cx 
\end{equation}

With the solution:
\begin{equation}
	x = e^{At} x_0 + \int_{0}^{t} e^{A(t-s)} B u(s) ds  
\end{equation}

\vspace{5pt}
\hrule
\vspace{5pt}

First we consider if a system is stable. 
\begin{equation}\label{eq1}
\Sigma_1 = \dot{x} = Ax, x(0) = x_0 \rightarrow x=e^{At} x_0
\end{equation}

Where we use the taylor expansion to find $ e^{At}  $
\begin{equation}
	e^{At} = \sum_{k=0}^{\infty}{\frac{(At)^{k}}{k!} }
\end{equation}

\subsection{Example}
\begin{equation}
A = \begin{bmatrix}
\lambda_1 & 0 \\
0 & \lambda_2
\end{bmatrix}
\end{equation}
\begin{equation}
	e^{At} = \sum_{k=0}^{\infty}{\frac{1}{k!} \begin{bmatrix}
	\lambda_1 t & 0 \\
	0 & \lambda_2 t
	\end{bmatrix}} = \sum_{k=0}^{\infty}{\frac{1}{k!} \begin{bmatrix}
	(\lambda_1t)^{k}  & 0 \\
	0 & (\lambda_2t)^{k}
	\end{bmatrix}}
\end{equation}
Note that It is only possible to do elementwise esponentials because $ A $ is Diagonal.

\begin{equation}
 = \sum_{k=0}^{\infty}{\begin{bmatrix}
		\sum_{k=0}^{\infty}{\frac{(\lambda_1t)^{k}}{k!}} & 0 \\
	0 & \sum_{k=0}^{\infty}{\frac{(\lambda_2t)^{k}}{k!}}
\end{bmatrix} } = \begin{bmatrix}
e^{\lambda_1t} & 0 \\
0 & e^{\lambda_1t} 
\end{bmatrix}
\end{equation}

And using (\ref{eq1})
\begin{equation}
x = e^{At} x_0 = \begin{bmatrix}
e^{\lambda_1t} x_{01} \\
e^{\lambda_2t} x_{02}
\end{bmatrix} 
\end{equation}

And if $ \lambda_1  < 0$ and $ \lambda_2 > 0$, the phase plot entails that it is not stable. 

\vspace{5pt}
\hrule
\vspace{5pt}
Since it is not always possible to do exponetials in (7), we set up a general ca, we set up a general case.

\begin{equation}
	M =  e^{At} \rightarrow m_{ij} = \sum_{i}^{}{\alpha_i t^{i} e^{\lambda_i t}} 
\end{equation}
Where $ \lambda_i \in \sigma(A) $ is the spectrum of $ A $, namely the set of eigenvaules of  $ A $.

\vspace{5pt}
\hrule
\vspace{5pt}
\begin{center}
\fbox{ \textbf{def:} The system $ \Sigma_1 $ is stable if for every $ x_0 $ $ x(t;x_0) \rightarrow 0 $ for $ t \rightarrow 0 $. }
\end{center}

Note: If  $ A $ is non-singular, then the only steady state vector is the null vector.

This is the same as positive definitness:
\begin{center}
	\fbox{ $ x^{T} B x > 0 \hspace{10pt} \forall x \neq 0 $ }
\end{center}
Where $ B $ has to be symmetric.\\ \textbf{Note} that this is easy to check, becuase that $ B $ has to have positive eigenvalues. Therefore  $ 0 < \lambda_i  \in \sigma(B) $ 

This entails the theorem:
\begin{center}
	\fbox{ \textbf{Theorem:} The system $\Sigma_1$ is stable iff  $ Re(\lambda) < 0 \hspace{10pt} \forall \lambda \in  \sigma(A) $
	}
\end{center}


\vspace{5pt}
\hrule
\vspace{5pt}
Sidenote on complex numbers and imaginary axis:

\begin{equation}
 e^{\lambda t} = e^{\alpha + i \beta)t} = e^{\alpha t} e^{i \beta t} 
\end{equation}

Where if $ \alpha < 0 $ the spiral is $ \rightarrow 0 $, and it has a small magnitude.

\vspace{5pt}
\hrule
\vspace{5pt}
\textbf{Relation to Kalman Decomposition} 

\begin{equation}
	\dot{\begin{bmatrix}
z_1 \\
z_2
\end{bmatrix}}  = \begin{bmatrix}
A_{11} & A_{12} \\
0 & A_{22}
\end{bmatrix} = \begin{bmatrix}
z_1 \\
z_2
\end{bmatrix} + \begin{bmatrix}
B_1 \\
0
\end{bmatrix} u
\end{equation}
Where

\begin{equation}
	\dot{z_1} = A_{11} z_1	 + A_{12} z_2 + B_1 u
\end{equation}
\begin{equation}\label{eq2}
	\dot{z_2} = A_{22} z_2
\end{equation}

Note that $ A_{12} $ does not really have an effect on the system, as the $ z_2 $ is dependent on (\ref{eq2}).

\vspace{5pt}
\hrule
\vspace{5pt}
\textbf{Introducing the Gain} 
\begin{equation}
\dot{x} = Ax + Bu
\end{equation}
\begin{equation}
u = Kx
\end{equation}
Where $ K \in \mathbb{R}^{m \times n} $  and this entails with substitution:
\begin{equation}
\dot{x} = Ax + Bu = (A+BK)x 
\end{equation}

Then we can choose $ K $ s.t.  $ \sigma(A + BK ) \subset \mathbb{C}_{-} $\\
\fbox{$\mathbb{C}_{-} = {\{ \lambda \in \mathbb{C}| Re(\lambda) < 0 \}}$}
Thus it is the set of all complex numbers with negative real parts. 

\section{Stabilizability}

Control system: whenever nothing is written on summation circles, it is always $ + $.

% \begin{center}
\begin{framed}
	\textbf{def:} The system $ \Sigma $ or the pair  $ (A,B) $ is stabilizable if there is a $ K $ s.t. $ \sigma(A + BK) \subset \mathbb{C}_{-} $ \vspace{10pt}\\ 
	\textbf{Then:} Let $ \Lambda = \{\lambda_1, ... , \lambda_n\} \subset \mathbb{C}$ with $\Lambda$ then $ \lambda_i^{*} \in \Lambda $. Then $ (A,B) $ is controllable then there is a $ K $ s.t.  $ \sigma(A+BK) = \Lambda $
\end{framed}
% \end{center}

\section{Reference Tracking}
% If we want the output $ y $ of the system to go to a value. 

Given a (constant) $ r $ find  $ u $ s.t. $y(t)  \rightarrow r $ as $ r \rightarrow \infty $

Note that $ \dot{x} \rightarrow 0 $ when steady state has reached. This entails that there is not dynamics at that point.

So:\
Given the state input pair $ (x',u') $ 
\begin{equation}
0 = Ax' + Bu'
\end{equation}
\begin{equation}
r = Cx'
\end{equation}


Let 
\begin{equation}
\begin{bmatrix}
0 \\
I
\end{bmatrix} r = \begin{bmatrix}
A & B \\
C & 0
\end{bmatrix} \begin{bmatrix}
x' \\
u'
\end{bmatrix} 
\end{equation}
\begin{equation}
\begin{bmatrix}
x' \\
u'
\end{bmatrix} r = \begin{bmatrix}
A & B \\
C & 0
\end{bmatrix}^{-1}  \begin{bmatrix}
0 \\
I'
\end{bmatrix} 
\end{equation}

You can only do reference tracking for inputs with the same number of inputs as outputs.  $ (n+p) \times (n+m), p = m $

\begin{equation}
= \begin{bmatrix}
N_x \\
N_u
\end{bmatrix} r 
\end{equation}

\vspace{5pt}
\hrule
\vspace{5pt}

So we want to design a control input $ u $ such that  $ y $ follows  $ r $.
We introduce a new variable $ z = x - x'$ and $ v = u - u' $ and whenever  $ z \rightarrow 0 $, $ x \rightarrow x' $ 

Lets start by finding $ \dot{z} $
\begin{equation}
\dot{z} = \dot{x} = Ax + Bu = Az + Ax' + Bv + Bu' = Az + Bv
\end{equation}
Hence we chose a $ v = kz $ s.t. $\sigma(A + Bk) \subset \mathbb{C}_{-}$ then:
\begin{equation}
	z(t) = x(t) - x' \rightarrow 0 \text{that is} x(t) \rightarrow x' \text{so} y(t) = C x(t) \rightarrow  C x' = r
\end{equation}

Do using:\\
$  v = kz \leftrightarrow u-u' = k(x-x') \leftrightarrow $
\begin{framed}
$  u = k(x-x') + u'\\ = k(x- N_x r) + N_u r \\ = kx + Nr \hspace{10pt} N = Nu - K N_x $
\end{framed}

\vspace{5pt}
\hrule
\vspace{5pt}
When the model is not precise enough, there will be some \textbf{error}, namely the steady state error.

\vspace{5pt}

\textbf{Side Note} we can use the place function in matlab to place poles. Note that it assumes positivity, so add a minus before it. 

For this we should use the integral action on the system. We introduce a new state:
\begin{equation}
	X_I = \int e \hspace{2pt} ds \hspace{10pt} e = y - r
\end{equation}

\begin{equation} \label{eq3}
\dot{x} = Ax + Bu
\end{equation}
\begin{equation}
	\dot{x_I} = e = y-r
\end{equation}
\begin{equation} \label{eq4}
y = Cx
\end{equation}

Combining (\ref{eq3}) - (\ref{eq4}) entails 
\begin{equation}
	\dot{\begin{bmatrix}
x \\
x_I
\end{bmatrix}} = \begin{bmatrix}
A & 0 \\
C & 0
\end{bmatrix} \begin{bmatrix}
x \\
x_I
\end{bmatrix} + \begin{bmatrix}
B \\
0
\end{bmatrix} u + \begin{bmatrix}
0 \\
-I
\end{bmatrix} r
\end{equation}

And by renaming the different terms:
\begin{equation}
	\dot{\overline{x}} = \overline{A}\overline{x} + \overline{B}u + G r
\end{equation}

and we do exactly the same as before:
\begin{equation}
z = \overline{x} - \begin{bmatrix}
x' \\
0
\end{bmatrix} \hspace{10pt} v = u - u'
\end{equation}
\begin{equation}
\dot{z} = \overline{A}\left(z + \begin{bmatrix}
x' \\
0
\end{bmatrix} \right) + \overline{B} (v + u') + Gr = \overline{A}z + \overline{B}v
\end{equation}

Hence
\begin{equation}
	v = \overline{K}z \hspace{5pt} u = \overline{k} \begin{bmatrix}
	x - x' \\
	x_I
	\end{bmatrix} + u' 
\end{equation}
\begin{equation}
	= \begin{bmatrix}
	k & k_I 
	\end{bmatrix} \begin{bmatrix}
	x - x' \\  
	x_I
	\end{bmatrix} 
\end{equation}
\begin{equation}
= k(x-x') + u' + k_I x_I
\end{equation}
\begin{equation}
= k (x - N_x r) + N_u r + k_I x_I
\end{equation}


Note that $ K $ has to be designed at once, you cannot design  $ K $ and  $ K_I $ seperately.


\vspace{5pt}
\hrule
\vspace{5pt}

\section{Linear Quadratic Regulator}

The way we choose a controller is by minimizing the control.

\begin{equation}
	min_{u} \int{0}^{T}{x^{*} Q x + u^{*} R u \hspace{2pt} dt}
\end{equation}
s.t. 
\begin{equation}
\dot{x} = Ax + Bu
\end{equation}

And the solution is:
\begin{equation}
 u = R^{-1}B^{*}Px   
\end{equation}
where the part before $ x $ is the gain K
where 
\begin{equation}
	T < \infty : \dot{P} = - Q - P A^{*} - AP + PBR^{-1}B^{*}P, \hspace{10pt} P(T) = 0   
\end{equation}

\begin{equation}
	T = \infty : 0 = - Q - P A^{*} - AP + PBR^{-1}B^{*}P
\end{equation}

and $ Q \geq 0 $ and  $ R >0 $ must be positive semidefinte and positive definite, respectively.

Here is the R identity and $ Q = \begin{bmatrix}
1 & 0 \\
0 & 10
\end{bmatrix} $
\begin{equation}
\int_{0}^{T}{x^{*} \begin{bmatrix}
1 & 0 \\
0 & 10
\end{bmatrix} x \hspace{5pt} dt } = \int_{0}^{T}{x_1^{2} +10 x_2^{2} dt}
\end{equation}


The control law will try to control the second part a lot quicker than the first part.

$ P $ is called the lapynov matrix, and it is widely used in control. 

When we do reference tracking:
\begin{equation}
	(y-r)^{*}Q(y-r) 
\end{equation}

\begin{framed}
\textbf{Then:} The system $ \dot{x} = Ax $ is stable iff $ \exists P > 0 $ s.t.  $ P A ^{*} + A P < 0 $ 	
\end{framed}

Think of it as the energy of the system.
$ x^{T} P x $
And it is used in the section on constraints.

\end{document}
