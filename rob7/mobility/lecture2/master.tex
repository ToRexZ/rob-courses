\documentclass[a4paper]{article}

\usepackage[utf8]{inputenc}
\usepackage[T1]{fontenc}
\usepackage{textcomp}
\usepackage[english]{babel}
\usepackage{amsmath, amssymb}
\usepackage{import}
\usepackage{pdfpages}
\usepackage{transparent}
\usepackage{xcolor}

% Remove paragraph indentation.
\setlength{\parindent}{0pt}

% Figure support
\usepackage{xifthen}
\pdfminorversion=7

\newcommand{\incfig}[2][1]{%
    \def\svgwidth{#1\columnwidth}
    \import{./figures/}{#2.pdf_tex}
}

\pdfsuppresswarningpagegroup=1

\title{Robot Mobility: Lecture 2 \\
	\large Lecture Notes}
\author{Victor Risager}

\begin{document}
\maketitle

\section{Recap}

A system of first order ODE's

\begin{equation}\label{eq1}
\dot{z} = f(z,v)
\end{equation}
\[
\overline{y} = h(x)
\]

\subsection{Linearization}

where $ \dot{z} $ is a hidden state. Today we will only look at (\ref{eq1})\\
First we look at the equilibrium points $ f(\overline{z}, \overline{v}) $ and we end up with a system $ \dot{x} = Ax + Bu $. 

\hfill \fbox{$ A $ is the jacobian of $ f $ with respect to  $ z $ of ( $ \overline{z}, \overline{v}) $ }


\hfill \fbox{\textbf{Note:}  There always $ n $ states,  $ m $ inputs and  $ p $ outputs.} 


Remember the left side by: MOI * acceleration (Newtons law for rotating bodies)
\begin{equation}
ml^{2} \ddot{\theta} =  mgl sin \theta - \beta \dot{\theta} + m l cos \theta \ddot{x} 
\end{equation}


First we isolate $ \ddot{\theta}  $ (divide through with $ ml^{2}  $ (note that the constants are simplified to $ a,b,c $
\begin{equation} \label{eq2}
\ddot{\theta} =  a sin \theta - b \dot{\theta} + c cos \theta u
\end{equation}


$ \Downarrow $

We write up differentials up untill the one below the highest order of derivatives. in this case it is up to velocity, since (\ref{eq2}) uses acceleration.
\[
z_1 = \theta, z_2 = \dot{z}
\] 
\[
	\dot{z_1} = z_2, z_2 = a sin z_1 - b z_2 + c cos z_1 
\] 
The above two equations is the definition of $ \dot{z} f(z,v) $

\subsection{Equlibrium point}
Now we want to find the equilibrium point. $ 0 = f(\overline{z}, \overline{v}) $
Emmidiately we can see that the velocities are 0, this means that $ z_2 = \dot{z_1} = 0 $.  Thus 
 \[
0 = a sin \overline{z_1} + c cos \overline{z_1} \overline{v}
\] 

since we want the pendulumm to be upright, where $ \theta = 0 $. 
\[
z_1 = 0 \rightarrow \overline{v} = 0
\] 

This entails that we want to find the Jacobian at 0

\hfill \fbox{The first row of the jacobian is the gradient of $ \dot{z_1} = z_2 $}

\begin{equation} \label{eq3}
\dot{x} = \begin{bmatrix}
0 & 1 \\
a & -b
\end{bmatrix}x + \begin{bmatrix}
0 \\
0
\end{bmatrix} u
\end{equation}

\[
x = z - \overline{z} = z
\] 
\[
u = v - \overline{v} = v
\] 

\section{Controlability}
\begin{equation}
\Sigma: \dot{x} = Ax + Bu, x(0) = x_0, y = Cx
\end{equation}
where $ x(0) = x_0 $ is the initial condition

We can use \\
TODO intdef expands at in.
 \begin{equation}
	 x = e^{At}x_0 + \int_{0}^{t}e^{A(t-\tau)} Bu(\tau) d\tau
\end{equation}


We have the following definition:

\vspace{5pt}

\textbf{def:} 

The system $\Sigma$, or the pair $ (A,B)$, is controllable (at time $ T $) if for every  $ (x_0,x_1) $ there exist a $ u \in U $

the notation is:
\fbox{$ x = x(t) = x(t, \overline{x}, u) $}
where $ \overline{x} $ is the state which i have to be in!



\begin{equation}
x_0 = x(0; x_0, u)
\end{equation}
\begin{equation}
x_1 = x(T; x_0, u)
\end{equation}

\begin{figure}[ht]
    \centering
    \incfig[1]{reachability}
    \caption{reachability}
    \label{fig:reachability}
\end{figure}

\newpage


\section{Reachability}
% \fbox{ \textbf{Note:} Reachability and controlability are the same in Continuous time, not in dicsrete time  }

Reachable subspace:
\begin{equation}
W_T = { \int_{0}^{T} e^{A(T-\tau)} Bu(\tau) d\tau | u \in U} \subseteq \mathbb{R}^{n} 
\end{equation}


\textbf{def:} 
$ \Sigma $ is reachable if $  W_T = \mathbb{R}^{n} $ \\
This entails that $ W_T = Range[A|B]$ \\
Where the reachability matrix: $ [A|B] = [B AB Ab^{2} \cdots A^{n-1}B] \in \mathbb{R}^{n \times nm}  $

\hfill \fbox{$ \Sigma $ is reachable $\leftrightarrow $ $ \Sigma $ is controlable}

\vspace{5pt}
\hrule
\vspace{5pt}
\textbf{Pendulum} 


\begin{equation}
	[A|B] = \begin{bmatrix}
	0 & c \\
	c & -bc
	\end{bmatrix}
\end{equation}
hence is controlable since it has full rank. It looses rank if $ c = 0$ so the condition is:   (if $ c \neq 0) $

\textbf{Example:}  
$ a = 2, b = c = 1 $.
Now based on (\ref{eq3})
\[
\dot{x} = \begin{bmatrix}
0 & 1 \\
2 & -1
\end{bmatrix}x + \begin{bmatrix}
0 \\
1
\end{bmatrix} u
\] 


We compute the eigenvalues:
\[
v_1 = \begin{bmatrix}
1 \\
1
\end{bmatrix}, \lambda_1 = 1 \rightarrow  Av_1 = \lambda_1 v_1
\] 
\[
v_2 = \begin{bmatrix}
-1 \\
2
\end{bmatrix}, \lambda_2 = -2 \rightarrow  Av_2 = \lambda_2 v_2
\] 

This system is unstable by nature. 

\vspace{5pt}

If you shut of the control $ \dot{x} = Ax $, and give it some initial condition. Then Figure \ref{fig:eigenspace} tells me how the system will behave. Only a few initial conditions, will bring the system to 2, and these are along the eigenvectors.

\begin{figure}[ht]
    \centering
    \incfig[1]{eigenspace}
    \caption{Eigenspace}
    \label{fig:eigenspace}
\end{figure}


Note that the straight arrows are the subspace spanned by the eigenvectors $ v_1 $ and $ v_2 $


\vspace{5pt}
\hrule
\vspace{5pt}

\section{Kalman decomposition}
What happens if we loose controlability: \\
i.e. if $ Rank[A|B] = l < n $ then there exists a matrix $ P \in \mathbb{R}^{n \times n } (z = Px)  $ s.t. 
 \[
	 \begin{gathered}
\dot{z} = P \dot{x} = P A x + P B u = P A P^{-1} z + P B u \\ = \begin{bmatrix}
A_{11} & A_{12} \\
0 & A_{22}
\end{bmatrix} z + \begin{bmatrix}
B_1 \\
0
\end{bmatrix} u
	 \end{gathered}
\] 
with $ (A_{11}, B_1) $ controlable. Moreover $ P = [e_1 \cdots e_l  e_{l+1} \cdots e_n]^{-1} $ \\ with $ Span\{e_1, \cdots, e_l \} = Range[A|B]$ and $ Span\{e_1, \cdots e_l\} = \mathbb{R}^{n} $

If we constrain the environment of reachability, then we can still control it. It is still reachable in the $ l$-dimensional subspace.


TODO: Add split and gathered to equation environments
\begin{equation}
	\begin{split}
	\begin{gathered}
		\dot{z_1} = A_{11} z_1 + A_{12} z_2 + B_1 u \hspace{10pt} A_{11} \in \mathbb{R}^{l \times l}  \\
	\dot{z_2} = A_{22} z_2
	\end{gathered}
	\end{split}
\end{equation}


Note that $ \dot{z_2} $ will either approach 0 or $ \infty $

\vspace{5pt}
\hrule
\vspace{5pt}
$ B = \begin{bmatrix}
0 \\
1
\end{bmatrix}  \rightarrow B = \begin{bmatrix}
-1 \\
2
\end{bmatrix} v_2 $ 
\[
	[A|B] = [B AB] = [v_2 \lambda_2v_2] = \begin{bmatrix}
	-1 & 2 \\
	2 & -4
\end{bmatrix} \hspace{10 pt} Rank[A|B] = 1 < 2
\] 

Lets do an orthogonal basis to $ v_2 $:
\begin{equation}\label{eq4}
P = \begin{bmatrix}
-1 & -2 \\
2 & -1
\end{bmatrix}^{-1} 
\end{equation} 
\vspace{5pt}

Now we compute $ P A P^{-1} $

(\ref{eq4}) implies that:
\begin{equation} \label{eq5}
\begin{split}
\begin{gathered}
\dot{z} = \begin{bmatrix}
-2 & -1 \\
0 & 1
\end{bmatrix}z + \begin{bmatrix}
1 \\
0
\end{bmatrix} u \\
\dot{z_1} =-2 z_1-z_2 + u \\
\dot{z_2} = z_2 \rightarrow z_2 = e^{t}z_{20}
\end{gathered}
\end{split}
\end{equation}
where $ z_{20} $ is an initial condition of $ z_2 $. The value of $ \dot{z_2} \rightarrow \infty $ 

In order to reach controllability, we need the points to be on the subspaces spanned by the eigenvectors. These are the controlable subspaces. 


next we do $ v_1 $
\[
B = \begin{bmatrix}
0 \\
1
\end{bmatrix} \rightarrow B = \begin{bmatrix}
1 \\
1
\end{bmatrix} = v_1 \rightarrow \dot{z} = \begin{bmatrix}
1 & -1 \\
0 & -2
\end{bmatrix}z + \begin{bmatrix}
1 \\
0
\end{bmatrix}
\] 
note that, the $ -2 $, results in the $ t $ in (\ref{eq5}) is negative, thus it is controlable.



\end{document}
