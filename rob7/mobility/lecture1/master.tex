\documentclass[a4paper]{article}

\usepackage[utf8]{inputenc}
\usepackage[T1]{fontenc}
\usepackage{textcomp}
\usepackage[english]{babel}
\usepackage{amsmath, amssymb}
\usepackage{import}
\usepackage{pdfpages}
\usepackage{transparent}
\usepackage{xcolor}

\setlength{\parindent}{0pt}

% figure support
\usepackage{xifthen}
\pdfminorversion=7

\newcommand{\incfig}[2][1]{%
    \def\svgwidth{#1\columnwidth}
    \import{./figures/}{#2.pdf_tex}
}

\pdfsuppresswarningpagegroup=1

\title{Mobility: Lecture 1 \\
	\large Lecture Notes}
\author{Victor Risager}

\begin{document}
\maketitle

\section{Introduction}
4 hours on next 6 mondays. 
Shahab will tell us on ordinary system theory after John.

We are going to try to make a turtlebot move in the end of the course. We need to use the theory. 
The exercises and the turtlebot will be taken care off by Mirhan.


\vspace{5pt}
\hrule
\vspace{5pt}
\section{Input-Output systems}
The standard differential equation. 
 \[
\Sigma: 
\dot{x} = f(x,u) \newline
y = h(x)
\] 


where $f: D_x \times D_u \rightarrow \mathbb{R}^n, D_x,  \subset \mathbb{R}^n, D_u \subset \mathbb{R}^m$


%\begin{figure}[ht]
%    \centering
%    \incfig[1]{iosystem}
%    \caption{ioSystem}
%    \label{fig:iosystem}
%\end{figure}


\vspace{5pt}
\hrule
\vspace{5pt}
\section{First Exercise: Cart Pendulumn}
%\begin{figure}[ht]
%    \centering
%    \incfig[1]{cartpendulum}
%    \caption{cartPendulum}
%    \label{fig:cartpendulum}
%\end{figure}




\subsection{Lagrange (d'Alembert)}
$L = T - U$

Where T is the kinetic energy, and U is the potential energy.

The coordinates:
You would need 2 coordinates to describe this cart system.

 
\begin{equation}
q = \begin{bmatrix}
x \\
\theta 
\end{bmatrix}
\end{equation}

\begin{equation}
r_c = \begin{bmatrix}
x \\
0 
\end{bmatrix}
\end{equation}


\begin{equation}
v_c = \dot{r_c} = \begin{bmatrix}
\dot{x} \\
0
\end{bmatrix}
\end{equation}



Where $v_c^2 = v_c \cdot v_c$ is the inner product. It gives a scalar. 
$= v_c^T v_c = \dot{x}^2$


\[
r_p = \begin{bmatrix}
x_p  \\
y_p 
\end{bmatrix} = \begin{bmatrix}
x - l\cdot sin(\theta) \\
l\cdot cos(\theta)
\end{bmatrix}
\] 

\begin{equation}
\begin{split}
	v_p = \begin{bmatrix}
	\dot{x}-l\cdot cos(\theta) \dot{\theta} \\
	- l \cdot sin(\theta) \dot{\theta}
	\end{bmatrix} 
	v_p^2 = v_p\cdot v_p = v_p^T v_p = (\dot{x} - l\cdot cos(\theta\dot{\theta})^2 + (l\cdot sin(\theta \dot{\theta})) \\
	= \dot{x}^2 + l^2 cos^2(\theta \cdot \dot{\theta}^2) - 2 \dot{x} l cos(\theta) \dot{\theta} + l^2 sin^2(\theta \dot{\theta}^2 \\
	= \dot{x}^2 + l^2 \dot{\theta}^2 - 2\dot{x} l cos(\theta \dot{\theta})
\end{split}
\end{equation}



$ \theta + 2 $


$ \mathcal{L}\{sin(t)\} = \frac{1}{s + 1} $


\subsubsection{Kinetic energy: $T$}
The kinetic energy of the cart, $T_c$, and the pendulum, $ T_p $

\vspace{5pt}
\textbf{Cart} 
\begin{equation}
T_c = \frac{1}{2}m_c v_c^2 = \frac{1}{2}m_c \dot{x}^2
\end{equation}


\textbf{Pendulum} 
 \begin{equation}
 T_p = \frac{1}{2} m_p v_p^2 = \frac{1}{2} m_p (\dot{x}^2 + l^2 \dot{\theta}^2 - 2\dot{x}l cos(\theta \dot{\theta})
\end{equation}
\begin{equation}
 T_c = \frac{1}{2} m_c v_c^{2} 
\end{equation}


\vspace{5pt}

The Total kinetic energy $ T = T_c + T_p $:
\begin{equation}
	T = \frac{1}{2} m_c \dot{x}^{2} +\frac{1}{2} m_p \dot{\theta}^{2} - m_pl^{2} \dot{\theta }^{2} - m_p \dot{x} l cos \theta \dot{\theta }
\end{equation}

and $ m = m_c + m_p $

\subsubsection{The Potential Energy}

\textbf{Cart} \\
The potential energy of the cart is 0, since it is constrained in the y direction. The potential energy is given relative to some point, and since we just assume the origin in at the same height as the cart, the potential energy $ u_c = 0 $  

\[
 u_c = m_c g h_c = m_c g 0 = 0 \hfill
\] 

\textbf{Pendulum} 
\[
	u_p = m_p g  h_p = m_p g l cos \theta 
\] 

Total potential energy
\begin{equation}
u = u_c + u_p = u_p = m_p g l cos \theta
\end{equation}

From this we now have the equations of motion (EOM), which we can use to derive the lagrange equation:

\begin{equation}
\frac{d}{dt} \frac{\partial \mathcal{L}}{\dot{q}} - \frac{\partial \mathcal{L}}{q}
\end{equation}
Where $  \frac{\partial \mathcal{L}}{q} $ is the conservative forces. \vspace{5pt} \\
\textbf{Conservative forces} are forces that do not change the energy that is going in or out of the system. \vspace{5pt} \\
\textbf{Non-conservative forces} change the energy of the system, e.g. friction.  \fbox{$ f_r = -\nabla u(r) $}


\textbf{Note:} That the lagrange equation has 4 independent variables, since $ q = \begin{bmatrix}
x \\
\theta
\end{bmatrix} $, thus $ \mathcal{L}(x, \dot{x}, \theta, \dot{\theta }) $. 

 
If we start to derive the lagrange equation w.r.t. $ q $, then we get:

\[
x: \frac{\partial \mathcal{L} }{\partial x} = 0
\] 
\[
\dot{x}: \frac{\partial \mathcal{L} }{\partial \dot{x}} = m \dot{x} - m_p l cos \theta \dot{\theta}
\] 
\[
\frac{d}{dt} \frac{\partial \mathcal{L} }{\partial \dot{x}} = m \ddot{x} - m_p l sin \theta \dot{\theta}^{2} - m_p l cos \theta \ddot{\theta} 
\] 

In general we have \hfill \fbox{$ \mathcal{L} = T - U $}

\begin{equation}
	\mathcal{L} = \frac{1}{2} m_c \dot{x}^{2} + \frac{1}{2} m_p \dot{\theta}^{2} -m_p l^{2} \dot{\theta}^{2} - m_p \dot{x} l cos \theta \dot{\theta} - m_p g l cos \theta
\end{equation}

Where the last term is $ U $

This results in the following equations of motion:
\begin{equation}
	m \ddot{x} + m_p l sin \theta \dot{\theta}^{2} - m_p l cos \theta \ddot{\theta} = - \alpha_x \dot{x} + u
\end{equation}

Where the right term is any non conservative force; in this case it is viscous friction, with the friction coefficient $ \alpha_x $










\end{document}
