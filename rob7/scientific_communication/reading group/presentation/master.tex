\documentclass[a4paper]{article}

\usepackage[utf8]{inputenc}
\usepackage[T1]{fontenc}
\usepackage{textcomp}
\usepackage[english]{babel}
\usepackage{amsmath, amssymb}
\usepackage{import}
\usepackage{pdfpages}
\usepackage{transparent}
\usepackage{xcolor}
\usepackage{framed}

% Remove paragraph indentation.
\setlength{\parindent}{0pt}

% Figure support
\usepackage{xifthen}
\pdfminorversion=7

\newcommand{\incfig}[2][1]{%
    \def\svgwidth{#1\columnwidth}
    \import{./figures/}{#2.pdf_tex}
}

\pdfsuppresswarningpagegroup=1

\title{Presentation: P4  \\
	\large Safe and Quasi-Optimal Autonomous Navigation in Environments with Convex Obstacles.}
\author{Victor Risager}

\begin{document}
\maketitle


\section{Test}
The planner is tested and compared to other planners like Djikstra's algrithm on a visibility tangent graph in 10 different Highly congested two-dimensional spaces. In each space 100 initial conditions are taken and the percentage of perfect mathcing paths with respect to the optimal Dijkstra planner is shown here. On e reason why it occasionally fails to take the optimal path, is because of the nested projections as seen in this figure.


\section{Benefits over Dijkstra}
The main benefits of this algorithm is that it
\begin{itemize}
	\item Solves the problem from a feedback control perspective, thus it can navigate in one go. 
	\item This is a closed-form solution, which is more suitable for realtime implementation.
	\item It is applicable in n dimensions. 
\end{itemize}


\section{Simulation}
The simulation is initially done in 2D and is compared to two other algorithms: Navigation function and seperating Hyperplane. The improventment in length of the paths generated can be seen in this table, where it is evident that there is an overall improvement. 

\section{Sensor based control strategy}
Implementation wise, when the world is considered unknown and only a Lidar is used, then some of the previous implications disapears. When using the lidar for navigation, then it is invariant of the nests generated by the undesired equilibria. 

\section{Gazebo simulation}
Again here the lidar is used again, but implemented on a turtlebot 3 and simulated in Gazebo using ROS noetic. As the Turtlebot is a non-holonomic differential drive robot, they propose a smooth switching between actual control inputs and linear and angular velocities which takes the dynamic limitations of the robot into consideration. This showed promising results as well, where no implications was observed. 

\section{Conclusion}
The algorihtm proposed achieves intermediary optimal obstacle avoidance, while the overall path is quasi-optimal but collision free. All of this is achived on the cost of a somewhat restrictive assumption on the shape of the obstacles, which have to be convex and reasonably curved. For the sensorbased version, problems of undesired equilibria does not occur, which allows it to navigate without any prior knowledge of the environment of sufficiently curved obstacles. 


\end{document}
