\documentclass[a4paper]{article}

\usepackage[utf8]{inputenc}
\usepackage[T1]{fontenc}
\usepackage{textcomp}
\usepackage[english]{babel}
\usepackage{amsmath, amssymb}
\usepackage{import}
\usepackage{pdfpages}
\usepackage{transparent}
\usepackage{xcolor}

% Remove paragraph indentation.
\setlength{\parindent}{0pt}

% Figure support
\usepackage{xifthen}
\pdfminorversion=7

\newcommand{\incfig}[2][1]{%
    \def\svgwidth{#1\columnwidth}
    \import{./figures/}{#2.pdf_tex}
}

\pdfsuppresswarningpagegroup=1

\title{Scientific Communication: Lecture 1 \\
	\large Lecture Notes}
\author{Victor Risager}

\begin{document}
\maketitle

\section{Introduction}
We are writing a paper on 7'th semester, and a couple of times on the 9'th semester.

\section{Writig Scientific papers}
It is not just writing, it is as much about presenting the material.

As a scientis you are supposed to ask questions, while in companies you are supposed to make solutions/products. 


\textbf{Outline} 
\begin{enumerate}
	\item Scientific publication
	\item SEMCON
	\item Structuring a scientific paper
	\item Summary
\end{enumerate}

\section{Why publish scientific results?}
Science is about uncovering the truth. \\
It is just as much a resource, where we have access to large peerreviewed databases with papers.

A document is called a publication, that can be submitted to a conference, journal or similar.

\subsection{Peer review}
Other scientists read the publication and critisize it, or come up with flaws, or point out if it has been done before. Then it might be rejected or accepted. Sometimes it needs a major revision.


\subsection{Publication}
Conferences are typically offer fast publications. Less than one year. 

\subsubsection{Paper}
Papers are shorter, than journals.
Shorter lifetime

Revises and edit it thoroughly.

\subsubsection{Journals}
Journals can have longer approval processes, and tougher rejection rate. 
'Live forever', they should not be outdated. \\
\textbf{Note} the scope of the publication is important. 

Most prestiguous journal is "Nature" and there is also one called "science" which is also very big.
They have a clear timeline. e.g. if any information gets disproved, then a new and updated version will get published, and then it will have the viable dates.
IEEE does also have high impact.


\section{SEMCON}
Acronym for a 7'th semester conference.
We will make a poster which are going to be put on the walls.

We will be divided up into smaller groups, which are cross field. 

We will meet again on november 27'th of november. Which is should at least be a draft. 

\textbf{Deadline for paper is November 27} 

The style should be \textit{ IEEE conference style} 
\textbf{Max 6 pages} (including everything, title, headline, citations)

If we do not have results yet, describe what the expected results are.

Review of papers for other groups, and give 1 page of feedback by no later than December 8.

Middle of December: Prepare slides and a poster.
Handin: Paper + Worksheets

SEMCOM is on December 19.
\subsubsection{Worksheets}

The worksheets is the report. Not as nicely written as the standard project report.


\section{Examination}
Can include the poster, scientific work, paper etc. 
otherwise it is essentially the same.

\section{Structuring a scientific paper}
The ABC's of science

\begin{itemize}
	\item \textbf{A}ccurate and \textbf{A}udience-adapted  
	\item \textbf{B}rief 
	\item \textbf{C}lear 
\end{itemize}

It is difficult to provide only the stuff that is needed.
Pay attention to that you are writing to somenone who may not be in the same field. 
\begin{itemize}
	\item Readable to people from other fields, (on the same level).
\end{itemize}

\textbf{Important} to state what is the contribution, the new results. What is the take home messages? 

\begin{itemize}
	\item Make tables and figures of interesting results to better convey the messages
	\item Make an outline - in your own style (but not this semester)
\end{itemize}



\subsection{Sections}


Sturecute to IMRaD (Introduction, Methods, Results and Discussion)\\
\textbf{Title, abstract, Paper}, each describe what is going on in the paper but in different levels of detail. 
\begin{itemize}
	\item \textbf{Title} 20 words. Most importantly that it describe what is going on in the paper.  
	\item \textbf{Authors}, affeliation (line of study, and institution)
	\item \textbf{Introduction}, the problem, what is known, what is not known and the objective.
	\item \textbf{Materials and methods} What we have done.
	\item \textbf{Results} What we found. 
	\item \textbf{Discussion}, discuss what was found out; backed up by data.
	\item Conclusions, can be combined with discussion.
	\item \textbf{Acknowledgements}. Thank involved people. People who contributed but are not appart of the writing team. 
	\item \textbf{References}  
	\item \textbf{Appendices} are not used often. 
\end{itemize}


\subsubsection{Introduction}
Should provide answers to the questions: \\
\textbf{WHY} is the topic of interest. Only a few lines, which really hook the reader. \\
\textbf{WHAT} is the background of the previous solutions, or related work. \\
\textbf{WHAT} is the background of potential solutions. \\
\textbf{WHAT} was attempted in the present effort. \\
\textbf{WHAT} will be presented in this paper. 

Read relevant papers on the subject, steal what we can and add to it. \\
Make a hypothesis - Which is always testable. \\
To find a hypothesis, ask a \textit{Research question.} \\\
This question gives a foundation for the hypothesis. e.g.: \textit{Do students who attend more lectures get bettter exam results?} 





\subsubsection{Methods and Materials}
Theory, models, basic equations, Block Diagrams, Algorithms, etc.
\textbf{ Do not mix methods and results.  Don't describe the robots hardware in the methods section!! }\\
Do all methods first, then all the results. not intertwined.

\subsubsection{Results}


Be critical in the results. What is observed, even not evident in the data directly.


\section{Checklist}
Do the scientific work before writing the paper.\\
Decide the author(s), publisher\\
Read your publisher's Instructions to Authors.\\
Brainstorm, make notes under the IMRaD headings \\
Make figures


Get started with the First Draft.








\end{document}
