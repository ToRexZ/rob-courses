\documentclass[a4paper]{article}

\usepackage[utf8]{inputenc}
\usepackage[T1]{fontenc}
\usepackage{textcomp}
\usepackage[english]{babel}
\usepackage{amsmath, amssymb}
\usepackage{import}
\usepackage{pdfpages}
\usepackage{transparent}
\usepackage{xcolor}
\usepackage{framed}
\usepackage[margin=2.5cm]{geometry}

% Remove paragraph indentation.
\setlength{\parindent}{0pt}

% Figure support
\usepackage{xifthen}
\pdfminorversion=7

\newcommand{\incfig}[2][1]{%
    \def\svgwidth{#1\columnwidth}
    \import{./figures/}{#2.pdf_tex}
}

\pdfsuppresswarningpagegroup=1

\title{Advanced Robotic Perception: Lecture 10 \\
	\large Lecture Notes}
\author{Victor Risager}

\begin{document}
\maketitle
\section{Camera-Robot calibration}
\begin{itemize}
	\item Eye-on-hand / eye-in-hand
	\item Eye-on-base / eye-to-hand
\end{itemize}

\subsection{Eye-on-hand - Theory}
We find transformations from the base of the robot tot the end effector, and a transform from the camera to the end-effector, and then from the object to the camera. 
This way we can compute the transformation between the object relative to the robot base. 

Then you can take multiple positions of the camera, and then we can equate them to one another. \\
The camera to the gripper transformation is unknown.

Rearranging the equation gives:
$ AX = XB $
Where
\begin{itemize}
	\item $ A $ is the robot transformations
	\item B is the camera transformations
\end{itemize}

\textbf{Seperable solutions} 
Estimate rotation, \textbf{and then} the translation

\textbf{Simultainously solutions} 
estimates and rotation and translation together.

\begin{enumerate}
	\item Mount camera
	\item Take images of checkerboards
	\item Calibrate camera $ \rightarrow  $ extract the extrinsic parameters.
\end{enumerate}

\textbf{Note:} the checkerboard must be fixed, but can be placed anywhere, because we use the relative pose between the measurements, and not the absolute positions. \\
\textbf{Note:} This method is only applicable if you do not know the actual transformation between the gripper and the camera mounted on the robot.

\subsection{Eye-on-base - Theory}
Here we place the checkerboard on the end-effector and move that around. Then we can estimate the camera to base transformation. \\
\textbf{Note:} that $ ^{g}T_t^{(1)} $  is the transformation from the calibration target (checkerboard) to the gripper. and it is the first measurement.

The base-gripper transformation is inverted compared to the other cameraplacement. 

\subsection{Other methods and variations}
\begin{itemize}
	\item Naive approach, $ \rightarrow  $ manual allignment 
	\item 3D printed camera mount where we know the actual position.
\end{itemize}

\end{document}
