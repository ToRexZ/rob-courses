\documentclass[a4paper]{article}

\usepackage[utf8]{inputenc}
\usepackage[T1]{fontenc}
\usepackage{textcomp}
\usepackage[english]{babel}
\usepackage{amsmath, amssymb}
\usepackage{import}
\usepackage{pdfpages}
\usepackage{transparent}
\usepackage{xcolor}
\usepackage{framed}

% Remove paragraph indentation.
\setlength{\parindent}{0pt}

% Figure support
\usepackage{xifthen}
\pdfminorversion=7

\newcommand{\incfig}[2][1]{%
    \def\svgwidth{#1\columnwidth}
    \import{./figures/}{#2.pdf_tex}
}

\pdfsuppresswarningpagegroup=1

\title{Advanced Robotic Perception: Lecture 7 \\
	\large Lecture Notes}
\author{Victor Risager}

\begin{document}
\maketitle
\section{Understanding neural networks}
The features are getting more holistic the further down the network we go. High level features combine shapes of the previous features. \\
Datasets:
\begin{itemize}
	\item PASCAL VOC
	\item MS COCO 
	\item LVIS
\end{itemize}

\subsection{Sliding window}
Instead of taking every window in the image, you start by identifying different regions, or blobs in the image that you can do CNN on.
This is called R-CNN.

Originally, run the entire Conv-Net and whenever we have classified the image, then we can run the last part of the network for each class/bounding box. 

RPN - Learnable region proposals


Single stage detectors are now better than multi stage.  This was up to 2017.

\section{Object detectors evaluation}
Comparing prediction bounding box to a ground truth bounding box, we can use the union of the two bounding boxes. 


mAP = Mean average precision




you can do learnable convolution to upsample the image. 












\end{document}
