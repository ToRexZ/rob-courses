\documentclass{article}


\usepackage{import}
\usepackage{pdfpages}
\usepackage{transparent}
\usepackage{xcolor}

\newcommand{\incfig}[2][1]{%
	\def\svgwidth{#1\columnwidth}
	\import{./figures/}{#2.pdf_tex}
}


\pdfsuppresswarningpagegroup=1

\title{Advanced Perception: Lecture 1 \\
	\large Intro, Convoluiotn and Edges
}
\author{Victor Risager}
\date{\today}



\begin{document}
\maketitle

\section{Intro}
The first 2 lectures are only us, then we are grouped with some other students who has to have the general introduction.  

There will be a miniproject after we are done in the lectures. Group based, and a presentation in front of the class. 


We need to program in OpenCV in either 
\begin{itemize}
	\item C++ 
	\item Python
\end{itemize}

\subsection{Exam}
20 minutes, oral exam.


\subsection{Projects of the institute}
\begin{itemize}
	\item Inspection of windmills
	\item Vision based slaughtering
	\item etc.
\end{itemize}

\section{Learning objectives}
\begin{itemize}
	\item Feature point algorithms
	\item Feature selection and reduction
	\item Motion analysis
	\item Tracking frameworks
	\item Advanced perception integrated into robotics
\end{itemize}

\subsection{Features}
We will look at more advanced features, detection and recocnition, tracking etc. 
used for tracking cars etc. 
Features can be used to create panorama images.

\section{Image correlation}
Find correlation in the input image

\[
	g(x,y) = h * f(x,y) = \sum_{-n}^{n}{x} \hfill
\] 

\section{Mean filter}
arrange the pixelvalues of the input image in the kernel, and do the average of it. Set the input of the center of the kernel in the output image.

\section{Convolution}
Convolutional neural networks do classical cross correlation.



The difference between convolution and correlation is in convolution the array is flipped and thenrun opposite. 

\section{Egdge detection}
\section{Canny and Sobel}
Canny is just sobel with more defined stuff. Canny is better.


\section{Laplacian}
Uses second order partial derivatives of the kernel.

Laplacian is sensitive to noise, so it is a good idea to use a smoothing kernel.


Can use fourier transform to filter out low frequency information or high frequency. Keep edges = high frequency.


\section{Difference of Gaussian}
Standard deviation is used to control the level of blurred ness. 



\section{Canny edge detector}
\begin{enumerate}
	\item Smooth the image
	\item Calculate the gradient and magnitude
	\item Nom-maximum suppression = keep the pixel value which has the highest probability of being an edge. 
\end{enumerate}

Find the pixel we want to keep by going along the gradient, and find the pixel with the highest value. 

Can potentially use 2 thresholds, to provide an interval.

\section{Harris corners}
Harris corners will be explained next week. Try today in the exercises.
Goal: Match two imagaes. Easier if they are partly overlapping.



\end{document}






