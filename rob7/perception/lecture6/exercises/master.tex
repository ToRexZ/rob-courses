\documentclass[a4paper]{article}

\usepackage[utf8]{inputenc}
\usepackage[T1]{fontenc}
\usepackage{textcomp}
\usepackage[english]{babel}
\usepackage{amsmath, amssymb}
\usepackage{import}
\usepackage{pdfpages}
\usepackage{transparent}
\usepackage{xcolor}

% Remove paragraph indentation.
\setlength{\parindent}{0pt}

% Figure support
\usepackage{xifthen}
\pdfminorversion=7

\newcommand{\incfig}[2][1]{%
    \def\svgwidth{#1\columnwidth}
    \import{./figures/}{#2.pdf_tex}
}

\pdfsuppresswarningpagegroup=1

\title{Exercises: Lecture 6 \\
	\large CNN Exercises}
\author{Victor Risager}

\begin{document}
\maketitle

\section{Exercise 1. Study the model that was created in the warm-up exercise. Try to answer the following:}
% \subsection{Look at the first Conv2D layer. Look at the documentation to find the meaning of the parameters: 32, kernel_size=(3,3), activation"relu"}

The first parameter is the \textbf{number of filters} \\
The second paremeter is the \textbf{size of the kernel} \\
The third parameter is the \textbf{type of activation function} 









\end{document}
