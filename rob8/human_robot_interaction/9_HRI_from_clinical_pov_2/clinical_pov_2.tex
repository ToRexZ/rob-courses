\documentclass[a4paper]{article}

\usepackage[utf8]{inputenc}
\usepackage[T1]{fontenc}
\usepackage{textcomp}
\usepackage[english]{babel}
\usepackage{amsmath, amssymb}
\usepackage{import}
\usepackage{pdfpages}
\usepackage{transparent}
\usepackage{xcolor}
\usepackage{framed}
\usepackage[margin=2.5cm]{geometry}
\usepackage{float}

% Remove paragraph indentation.
\setlength{\parindent}{0pt}

% Figure support
\usepackage{xifthen}
\pdfminorversion=7

\newcommand{\incfig}[2][1]{%
    \def\svgwidth{#1\columnwidth}
    \import{./figures/}{#2.pdf_tex}
}

\pdfsuppresswarningpagegroup=1

\title{Lecture 9: HRI from Clinical POV II  \\
	\large Human Robot Interaction}
\author{Victor Risager}

\begin{document}
\maketitle
\section{recap}

All of HRI fits into these categories. 
\begin{itemize}
	\item Prevention
	\item Diagnoisis
	\item Treatment
	\item Rehabilitation
\end{itemize}

\section{Exoskeleton}
Exoskeletons need to be fitted very precisely, and the patient may not be able to feel that it does not fit. One patient broke her angle while she was using a bipedal exoksleton.


\section{Human Robot Interaction}
The robot not only interacts with humans, but it can also affect the interaction between humans. Often times the HRI happens between the robot, patient and clinician.


\section{Ethics}
\begin{itemize}
	\item Morality? 
\end{itemize}

Most importantly ethics is a matter of perspective. How do you view a problem, and people may have a different perspectives. 

\subsection{Utilitarianism}
The greatest amount of good for the greatest number of persons.

\subsubsection{Quality Adjusted Life Year (QALY)}
A measure of how much quality that the patient would gain over some time. A measure between 0 and 1, where 0 is dead and 1 is perfectly healthy life. The main problem with this is how to evaluate what a procedure is worth?  WHO suggested that you can spend 1-3 GDP per capita, before it gets unprofitable. That is if we design a robotic device that is more expensive than this, then it would evidently render the procedure unprofitable. 


\subsection{Challenges for AI}
Trustworthy AI must provide:
\begin{enumerate}
	\item Transparency
	\item Creditablity
	\item Auditability
	\item Reliability
	\item Recoverability
\end{enumerate}

A wheelchair user had BCI and the wheelchair could act on its own. However if the patient fell asleep, the wheelchair would just take off and do its own thing.


\subsection{DNR - Do Not Recositate}
If the patient wants to not be revived, then the robout should not provide CPR. Respect the choices of the pateint.

Respect of autonomy. Do we trust the techonology.

Trust in technology. 

Human Autonomy and privacy.


\subsection{BCI}
Can you give consent through BCI? How do we accept it? Are there things that you cannot consent to? Can a patient consent that they decline further lifesupport.

\section{Ethics}
\subsection{Resaerch protocol}
How do you choose ethics. The Ethical committee can make choices based on the following data:
\begin{enumerate}
	\item Research
	\item Compile into human readable points for the participant
	\item Inform the patient $ \rightarrow $ One for each nationality
	\item Informed consent 
	\item Recruitment forms
	\item Add appendixes. Using e.g. surveys, or questionaires
	\item Application
	\item Protocol resume
\end{enumerate}


\subsubsection{Protocol / Resume}
There must always be a doctor who is medically responsible. 
What is the background and motivation for the experiment?
Define the purpose of the experiment

What methods are you going to use, how long the test is, and how many times you are going to use it. 
Describe any risks in the experiment
Describe the statistics that are to be used on the data. 

Show all the ethical considerations $ \rightarrow $ We can improve the recovery time.Ensure that we have considered all the previous considerations. 

How is the project funded. 
What is the timeline, and when are you done?

That was document 1.

\subsubsection{Ethical Protocol}
Fill out the information provided to the patient. This also includes rules like the GDPR rules. 
One information sheet to participants for every:
\begin{itemize}
	\item Experiment
	\item "Type" of participant, healty and a patient group to compare the two. 
	\item Expected language of participants
\end{itemize}

Questionaries does not count as an experiment.

\subsubsection{Process}
you need to send the data into the committe, and include money for it.




\section{Exercise}
Define what experiments that are necessary for the guy to be able to use the bipedal exoskeleton.




\end{document}
