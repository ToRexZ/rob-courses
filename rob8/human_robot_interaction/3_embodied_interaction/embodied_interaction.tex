\documentclass[a4paper]{article}

\usepackage[utf8]{inputenc}
\usepackage[T1]{fontenc}
\usepackage{textcomp}
\usepackage[english]{babel}
\usepackage{amsmath, amssymb}
\usepackage{import}
\usepackage{pdfpages}
\usepackage{transparent}
\usepackage{xcolor}
\usepackage{framed}
\usepackage[margin=2.5cm]{geometry}
\usepackage{float}

% Remove paragraph indentation.
\setlength{\parindent}{0pt}

% Figure support
\usepackage{xifthen}
\pdfminorversion=7

\newcommand{\incfig}[2][1]{%
    \def\svgwidth{#1\columnwidth}
    \import{./figures/}{#2.pdf_tex}
}

\pdfsuppresswarningpagegroup=1

\title{Lecture 3: Embodied Interaction  \\
	\large Human Robot Interaction}
\author{Victor Risager}

\begin{document}
\maketitle

\section{Embodied Interaction}
Example is \textit{affective} body movement of a robot. \\
Agents paradigm of AI:
\begin{itemize}
	\item Agents are systems that perceive their environment and act in it.
	\item 
\end{itemize}
General model of the cognitive part of the agent. 

\begin{itemize}
	\item Perceive
	\item Act
	\item Reason
\end{itemize}
A good example of this is a thermostat. Depending on the temperature of the room. we let the warn or cold water flow through. 

\section{Gestures}
Gestures complement what we say, and they may save time if used correctly (Explain where the buststop is without pointing). \\
Gestures may only have meaning in certain cultures or languages. 

There are multiple taxonomies of the gesutres. 
\begin{itemize}
	\item Speech linked
		\item Non speech linked. 
\end{itemize}

Gesture space is based around the center of the body. The are a periphary. 

There are multiple ways of doing the same pointing gestures. Depending on the how far out in the periphary the gesture is performed. 

\section{Expressivity Parameters}
Parameters:
\begin{itemize}
	\item Spatial volume (amplitude of the movement.)
	\item Speed (how quickly it is done)
	\item Energy (the level of overshoot)
	\item Fluidity (How continuous is it)
	\item Repitition (How often the movement is expressed)
\end{itemize}
3 phases of gestures (preparation, stroke and the retraction)


A gesture speed must match the speed of the sentence, otherwise there would be waiting time either on speech or by gesture. 

\section{Spatial Behaviour}
2 things to be aware of:
\begin{itemize}
	\item Proxemics.
	\item 
\end{itemize}
The size of robots has an effect on how close we allow the robot to get to us. pepper is like a child, so we automatically allow it to get closer. 

\section{Spatial orientation}
A conversation between two persons, and a 3 person joins them. This is called F-formations for static group interaction. 
\begin{itemize}
	\item r-space
	\item p-space
	\item o-space 
\end{itemize}
If the 2 initial persons that start facing each other, reforms the spatial oriantion to let the 3'rd person in. 

\section{Affective Interaction}
You can use emotions in interactions. Intrapersonal and interpersonal.

\subsection{Emotions}
Categorial models there are different levels, refer to the slide.
\begin{itemize}
	\item Anger
	\item Disgust
	\item Fear
	\item Enjoyment
	\item Sadness
	\item Suprise
\end{itemize}

It is also important to consider the dynamics of the emotions. E.g we may not know how you go from fear to sadness. What would happen?

PAD model (current way of determining dymanics)
\begin{itemize}
	\item 
\end{itemize}


\section{Emotion recognition}
Look at:
\begin{itemize}
	\item Facial Expressions
	\item Posture
	\item 
	\item 
	\item 
\end{itemize}
You can use emg to register emotional state. 


\end{document}
