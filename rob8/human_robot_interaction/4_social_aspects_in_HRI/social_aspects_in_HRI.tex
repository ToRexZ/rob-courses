\documentclass[a4paper]{article}

\usepackage[utf8]{inputenc}
\usepackage[T1]{fontenc}
\usepackage{textcomp}
\usepackage[english]{babel}
\usepackage{amsmath, amssymb}
\usepackage{import}
\usepackage{pdfpages}
\usepackage{transparent}
\usepackage{xcolor}
\usepackage{framed}
\usepackage[margin=2.5cm]{geometry}
\usepackage{float}

% Remove paragraph indentation.
\setlength{\parindent}{0pt}

% Figure support
\usepackage{xifthen}
\pdfminorversion=7

\newcommand{\incfig}[2][1]{%
    \def\svgwidth{#1\columnwidth}
    \import{./figures/}{#2.pdf_tex}
}
 
\pdfsuppresswarningpagegroup=1

\title{Lecture 4: Social aspects in HRI  \\
	\large Human Robot Interaction}
\author{Victor Risager}

\begin{document}
\maketitle

\section{Body movements}
\begin{itemize}
	\item Current robots are limited in their detection of facial expressions
		\begin{itemize}
			\item Age affects this
		\end{itemize}
	\item Determine how body movements depicts emotions. 
\end{itemize}

Cultures can affect how people show emotions, and if they show emotions. 

\section{The function of culture}
Defines how we interpret situations, attitude, goals, etc. It also provides behavior patterns. There are many ways of interpreting the word "culture"
\begin{itemize}
	\item Categorial (hall's) (widely used)
	\item Dimensional
	\item Brown/levinson
\end{itemize}

\subsection{Hall's Framework}
\begin{itemize}
	\item Space (Last lecture)
	\item Context
	\item Time
\end{itemize}

Context is how the communication is explicitness or maybe you infer sarcasm. \\
Monochronic time is cultures who follow time explicitly. 
Polychronic time does not necessarily use the clock, perhaps you start a meeting before everybody has shown up. They do not take clock time too seriosly.

\section{Systems of values: Hofstede}
This accounts for different peoples work cultures. They are distributed across 6 dimensions.

\begin{itemize}
	\item Power Distance Index (how much power is decregated)
	\item Individualism / collectivism (How individual do people be)
	\item Masculinity / feminity (Distribution )
	\item Uncertainty Avoidance Index 
	\item Long Term Orientation
	\item Indulgence-restraint
\end{itemize}
It is howerver descritptive, meaning that we still need data to describe reality a bit more. 


\section{Cultural Differences in communication Style}
\begin{itemize}
	\item Implicit (implicit communication $ \rightarrow $ Not so direct)
	\item Explicit (explicit communication $ \rightarrow $ Direct)
\end{itemize}


\section{Creating affective body movements across cultures}
Use Laban movement parameters to make the Nao robot epxress emotions. 

\section{Use decision trees}
To determine knocks on doors, you could infer different emotions of the different parameters. The most important paramter was the weight of the knock.

\section{Role of Trust in automation}
How can we change the trustworthyness calibration of trust in automation.

\subsection{Measuring trust}
This is usually done wit 40 questions questionairs. Good for ground truth, but it does not account for dynamically changing trust. 

\subsubsection{What do we need to do?}
\begin{itemize}
	\item Identify risk factors (What risks do exist)
	\item Manipulate the trust in princpled way. How can we bring people to low or high trust levels?
	\item Maping sensor data to trust score
	\item Assessment the trust levels
	\item Trust calibration
\end{itemize}

Looked at two different risk factors, and how does it affect the trust?
Is the robot in front or behind you. The riskier the task $ \rightarrow $ the less trust. 

Collaborative robot (you hold the paper, and the robot draws on the paper)
If we change the speed, the no matter the direction, the trust levels dip. Higher speed is lower trust, and lower trust development.

\subsection{Estimating trust }
You can use an IMU suit which can give specific movement outputs, which by running it through FCNN's gives a trust score, which can be plotted in the calibration square. 

\end{document}
